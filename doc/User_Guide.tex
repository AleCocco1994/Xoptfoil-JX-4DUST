%Compiling guide for XoptFoil
\documentclass[11pt]{article}

\usepackage{amssymb}
\usepackage{amsmath}
\usepackage{setspace}
\usepackage{graphicx}
\usepackage{subfig}
\usepackage{color}
\usepackage[left=1.2in, right=1in, top=1in, bottom=1in]{geometry}
\usepackage{verbatim}
\usepackage{hyperref}

\begin{document}

\title{XoptFoil User Guide}
\author{Daniel Prosser}
\date{\today}
\maketitle

\tableofcontents

\section{Compiling}

For convenience, precompiled versions of XoptFoil and its related tools are
located in the bin/x86-64 directory for Windows. These should work as long as
your CPU architecture is of x86-64 type, but to run the parallel version you will need to
compile it yourself. If your computer has more than one core, it will be greatly
beneficial to run the parallel code, so it's highly recommended that you compile it
yourself (plus it's not hard). The steps below explain how to compile the code.

\subsection{Linux}

Compiling is quite simple on Linux, but you will first need to install the compiler. The
GNU Fortran compiler, gfortran, is recommended and can be installed via the package
manager for your distro.  For example, on Ubuntu or Linux Mint, issue the command:

\begin{verbatim}
sudo apt-get install gfortran
\end{verbatim}

Next, navigate to the XoptFoil/bin directory.  A Makefile is included for compiling. Issue
the following command:

\begin{verbatim}
make -f Makefile_Linux_MacOSX
\end{verbatim}

This should compile all the source files and then link them into an executable.  If for
some reason Make isn't included on your system by default, install it using your package 
manager. If you invoke the command above without specifying a file name, it will 
automatically look
for a file called ``Makefile.''  Therefore, by renaming ``Makefile\_Linux\_MacOSX'' as
``Makefile,'' the make command can be simplified to:

\begin{verbatim}
make
\end{verbatim}

The Makefile contains various compiler options.  The default ones should work fine, but
you can feel free to change them if you want (look up gfortran compiler flags online).
XoptFoil is written so that it can run in parallel on computers with multiple
cores/threads.  This option is disabled in the precompiled binary, because it depends on
the pthreads system library.  You can enable it by compiling XoptFoil yourself.  If you
want to compile it yourself but don't want to use the parallel version, comment the flag
called OMP\_FLAG (by placing a \# character in front). Two executables are
created: xoptfoil, which is the main optimizer, and xfoil\_only, which can be used to run 
XFoil by itself for an airfoil at a given set of operating conditions.

\subsection{Windows}\label{sec:compile_windows}

The easiest way to get compilers for Windows is the MinGW project, which provides the GNU
compilers (gcc, g++, gfortran, etc.) and also some other tools such.  The main page for
MinGW is \url{mingw.org}.  The installation instructions are at
\url{http://mingw.org/wiki/Getting_Started}.  Follow the instructions under the section
``Graphical User Interface Installer'' first.  The process is fairly simple.  Once it is
installed, click Continue to open the MinGW installation manager.  The next step is to
select the components to install.  You will want to select mingw32-gcc-fortran 
(which will install gfortran, the Fortran
compiler used to build XoptFoil) and mingw32-base, which includes some other tools
which will be needed.  The dialog should look something like the one in Fig.
\ref{fig:installmingw1}(a) (note the selected items).

\begin{figure}[!h]
\centering
   \subfloat[Selecting Fortran compiler]{
     \includegraphics[width=0.95\textwidth]{manual_figs/selections.png}}

   \subfloat[pthreads library for parallel processing]{
     \includegraphics[width=0.95\textwidth]{manual_figs/pthreads_mingw.png}}
   \caption{MinGW Installation Manager showing compilers and libraries to be installed.
Note the screenshots in (a) and (b) are from different versions of the installer; hence
the slightly different appearance.}
\label{fig:installmingw1}
\end{figure}

In order to compile the parallel version on Windows, you also need to install the pthreads
library.  To do that, find the three packages labeled mingw32-pthreads-w32.  In the latest
version of the MinGW package installer, these are listed under the MinGW Libraries
grouping, as shown in Fig. \ref{fig:installmingw1}(b).  Mark each of these for
installation as well as the Fortran compiler. After making those selections, select
\textit{Apply Changes} under the Installation entry on the menu bar.  The installer will 
download and install the compilers you need.

At this point, the needed compilers are installed.  The next thing you need to do is to
make sure your computer can find them.  Follow the instructions under ``Environment
Settings'' on the MinGW Getting Started web page.  You will be modifying the Path
system environment variable by appending ``;C:$\backslash$MinGW$\backslash$bin''
to the end (unless you changed the installation directory to something other than
``C:$\backslash$MinGW'' when you ran the intallation manager).

The next step is to compile XoptFoil. This is process is automated using a program called
Make.  On MinGW, this tool is called mingw32-make.  To compile XoptFoil using Make, run the
following command from the command line inside the XoptFoil$\backslash$bin directory:

\begin{verbatim}
mingw32-make -f Makefile_Windows
\end{verbatim}

This should compile all the source files and then link them into an executable.  If
mingw32-make is not found, it is probably either because you didn't install mingw32-base
from the MinGW installation manager or because you didn't add the MinGW$\backslash$bin
directory to the Path system environment variable.  Check to make sure that mingw32-make
is in the MinGW$\backslash$bin directory first.  If it's not, then install it through the
MinGW installation manager.

If you invoke the command above without specifying a file name, it will automatically look
for a file called ``Makefile.''  Therefore, by renaming ``Makefile\_Windows'' as
``Makefile,'' the mingw32-make command can be simplified to:

\begin{verbatim}
mingw32-make
\end{verbatim}

The Makefile contains various compiler options.  The default ones should work fine, but
you can feel free to change them if you want (look up gfortran compiler flags online).
XoptFoil is written so that it can run in parallel on computers with multiple
cores/threads.  This option is disabled in the precompiled binary, because it depends on
the pthreads system library.  You can enable it by compiling XoptFoil yourself.  If you
want to compile it yourself but don't want to use the parallel version, comment the flag
called OMP\_FLAG (by placing a \# character in front). Two executables are
created: xoptfoil.exe, which is the main optimizer, and xfoil\_only.exe, which can be 
used to run XFoil by itself for an airfoil at a given set of operating conditions.

\subsection{Mac OSX}

In order to compile XoptFoil, you will need a compiler.  It is recommended to use the GNU
Fortran compiler, gfortran.  On Mac OSX, gfortran binaries are available at 
\url{http://gcc.gnu.org/wiki/GFortranBinaries#MacOS}.  Follow the instructions on that
website to install gfortran.  You will also need to install a program called Make.  Make
is available by installing Xcode developer tools from the App Store, which should
automatically place the make executable in a directory within your path. Once you have
gfortran and Make installed, the rest of the compiling process is identical to the Linux
instructions above.

\section{Running XoptFoil}\label{sec:running}

XoptFoil is a command-line based program.  Once you have compiled binaries for your
machine (see the Compiling Guide), running the program is very easy.  Simply open a
terminal (``Command Prompt'' in Windows) and navigate to the directory where the XoptFoil
executable resides.  For Windows, a simple batch script is included that will open a
command prompt in the current directory when you double-click on it.  To run XoptFoil,
simply type the following at the command prompt:

\begin{verbatim}
xoptfoil.exe
\end{verbatim}

\noindent if you are on Windows or:

\begin{verbatim}
./xoptfoil
\end{verbatim}

\noindent if you are on Linux or Mac OSX.  This will run XoptFoil in the current working
directory.  XoptFoil uses an input file called `inputs.txt,' which must also be in the
working directory.  If the input file is not in the working directory where XoptFoil is
run, the program will notify you that it can't find the file and then stop.  For that
reason, it's important that you follow the directions above to run the program to ensure
that the working directory will be set correctly.  If you double-click on the executable
instead of running it from the command prompt, it's possible that the working directory
won't be set to the directory where `inputs.txt' resides, and the program won't be able to
find it.

By default, XoptFoil writes the final optimized airfoil to an output file called
`optfoil.dat.'  However, you can change the name of both the input and output file, if
desired, by supplying command-line options (CLOs) when you run the program.  If only one 
CLO is present, it is
taken to be the name of the input file.  If two options are present, the first is taken to
be the name of the input file and the second as
the name of the output airfoil (the output file will have `.dat' appended to the name you
specify). If none are present, the program defaults to the
original file names.  So, for example, if you are running on Windows and you want to
use an input file called `case1.txt' and have the final optimized airfoil written to a
file called 'case1\_coordinates.dat,' you would invoke the command in this way:

\begin{verbatim}
xoptfoil.exe case1.txt case1_coordinates
\end{verbatim}

\section{Input file}

XoptFoil reads a Fortran namelist input file called `inputs.txt' (unless a different name
is specified as a CLO, as described above).  This file stores
conditions for optimization, such as the optimization type, parameterization settings,
aerodynamic operating conditions and constraints, the seed airfoil, XFoil options, etc.
There are a number of categories of inputs, each of which is stored in a separate
namelist.  A Fortran namelist is formatted as follows:

\begin{verbatim}
&namelist_title
  input1 = value1          %Comments are preceeded by a percent sign
  input2 = value2
  textinput1 = 'string1'
  etc.
/
\end{verbatim}

Each namelist begins with a `\&' character followed immediately by the title, and it ends with 
a '/' character.  Variables for each namelist are then listed with their respective
values.  To change an input, simply change the value to the right of the `=' sign.  Note
that text-based inputs (for example, the name of the seed airfoil file) need to be
surrounded by single or double quotes.  The inputs for each of the XoptFoil namelists,
stored in `inputs.txt,' are explained in the following sections. Sample input files are
included with XoptFoil.

\subsection{optimization\_options namelist}

In this namelist, high-level optimization and parameterization parameters are set.  Each
of the inputs is described in the list below.

\begin{enumerate}
\item{\textbf{search\_type}: May be `global', `local', or `global\_and\_local'.
A global search is an expensive approach like particle swarm or a genetic algorithm which
includes many potential designs to attempt to converge on the global best solution.  A
local search is one that investigates the design space in the vicinity of the initial
design (the seed airfoil), looking for improvements.  A local search is much faster than
a global search, but it also usually does not find as good a final solution.  If this
input is set to `global\_and\_local', a global search is first performed, followed by a
local search.}
\item{\textbf{global\_search}: Optimization algorithm for the global search.  Currently,
the only available option is `particle\_swarm'.}
\item{\textbf{local\_search}: Optimization algorithm for the local search.  Currently, the
only available option is `simplex'.}
\item{\textbf{seed\_airfoil}: Defines how to set the seed airfoil.  Available options are
(1) `from\_file', whereby the seed airfoil is read from a file, or (2) `four\_digit',
whereby the seed airfoil is set by a four-digit NACA code.}
\item{\textbf{airfoil\_file}: File name for the seed airfoil, which is used if
seed\_airfoil = `from\_file'.  The file must be formatted in XFoil format; it may have a
label on the first line (though it does not need to) and then the coordinates must be
arranged in two columns for x and y, forming a closed loop beginning and ending at the
trailing edge. There should be no blank lines, and you should also not include the number
of points or any other information besides a header and then the data.  If the file is not 
in the working directory, you must include the relative path.}
\item{\textbf{naca\_digits}: Text string specifying four NACA digits to define the seed
airfoil, which is used if seed\_airfoil = `four\_digit'.  The first digit is for max camber,
as a percentage of the chord length; the second digit is for location of max camber, in
tenths of the chord length, and the last two digits are for max thickness, as a percentage
of chord length.}
\item{\textbf{shape\_functions}: Identifier for the shape functions used to parameterize
airfoils in the optimization.  This may be `naca' or `hicks-henne'.  A user-defined number
of these shape functions are added to the top and bottom surfaces of the seed airfoil to
create new shapes.  `naca' functions are a family of functions which, when combined in a
weighted sum, can reproduce many of the NACA airfoils, including four-digit and transonic
airfoils.  `hicks-henne' are a more general class of functions which place a ``bump'' of
variable width and location on the airfoil surface.  Each Hicks-Henne shape function,
therefore, has a strength and also a width and location, whereas each NACA function only
has a strength.}
\item{\textbf{nfunctions\_top}: The number of shape functions used to parameterize the top
surface of the airfoil.  Note that if you are using Hicks-Henne functions, 1/3 the
number of shape functions will give the same number of design variables and a similar range of
possible shapes as the NACA functions.  For example, you may use 4 Hicks-Henne functions
instead of 12 NACA functions to have the same number of total design variables.}
\item{\textbf{nfunctions\_bot}: The number of shape functions used to parameterize the
bottom surface of the airfoil.  See the note on nfunction\_top regarding differences
between NACA and Hicks-Henne functions.}
\item{\textbf{initial\_perturb}: The maximum strength of the shape functions used when
generating initial designs.}
\item{\textbf{write\_designs}: Whether to write airfoil coordinates and polars to files
during the optimization.  See section \ref{sec:output_files} for more information on the
output files generated.}
\end{enumerate}

\subsection{operating\_conditions namelist}

This namelist specifies aerodynamic operating conditions over which to optimize the
airfoil.  These are the operating points at which each potential design will be evaluated,
including angle of attack or lift coefficient, Reynolds number, Mach number, and
relative weighting of each operating point.

\begin{enumerate}
\item{\textbf{noppoint}: The total number of operating points at which to evaluate
potential designs.  A maximum of 30 is allowed.  Including more operating points makes the
final design more robust.  With only one or two operating points, the final airfoil will
probably perform quite well at those conditions but do poorly elsewhere.  Including more
operating points is more expensive, however, because XFoil must be run for each airfoil at
each operating point.  The inputs below should be specified for each operating point.}
\item{\textbf{use\_flap}: Whether flap deflections will be applied.}
\item{\textbf{x\_flap}: Flap hinge point in x (between 0 and 1). Only used if use\_flap is
enabled.}
\item{\textbf{y\_flap}: Flap hinge point in y. Only used if use\_flap is enabled.}
\item{\textbf{op\_mode(\#)} (where \# is an integer ranging from 1 to noppoint): Tells
XFoil whether the angle of attack or lift coefficient is being specified for this
operating point.  May be `spec-al' for specified angle of attack or `spec-cl' for
specified lift coefficient. `spec-cl' is usually more useful, because generally the lift
coefficient is known as a function of the aircraft weight and flight speed during design.}
\item{\textbf{op\_point(\#)} (where \# is an integer ranging from 1 to noppoint):
Specifies either the angle of attack (in degrees) or the lift coefficient for this
operating point, depending on whether op\_mode is `spec-al' or `spec-cl' for this
operating point.  Note that XFoil calculations are not accurate above stall, so do not
specify too large a lift coefficient or angle of attack.}
\item{\textbf{optimization\_type(\#)}: (where \# is an integer ranging from 1 to
noppoint): Specifies the optimization objective for this operating point. Current options 
include `min-drag' (to minimize drag), `max-glide' (to maximize 
glide slope), `min-sink' (to minimize sink rate), or `max-lift' (to maximize lift at a
specified angle of attack).}
\item{\textbf{reynolds(\#)} (where \# is an integer ranging from 1 to noppoint): Specifies
the Reynolds number based on chord for this operating point.}
\item{\textbf{mach(\#)} (where \# is an integer ranging from 1 to noppoint): Specifies
the Mach number for this operating point.  Note that XFoil calculations are not accurate
at or above the transonic regime, so do not specify too high a Mach number.}
\item{\textbf{flap\_degrees(\#)} (where \# is an integer ranging from 1 to noppoint):
Specifies the flap deflection for this operating point, in degrees, with a positive
deflection being in the downward direction. Only used if use\_flap is enabled.}
\item{\textbf{weighting(\#)} (where \# is an integer ranging from 1 to noppoint): Specifies
the weighting for this operating point.  Operating points with a higher weighting are
given more importance by the optimizer.  Note that the weightings are automatically 
normalized by the optimizer so that they add up to 1.}
\end{enumerate}

\subsection{constraints namelist}

Specifies aerodynamic and geometric constraints for the optimization.  Constraints are
implemented in the optimizer by assigning a large penalty value whenever a constraint is
violated.  Note that in addition to the user-specified constraints, there are additional
constraints in the code to ensure convergence of XFoil viscous calculations and to prevent
unusual designs which XFoil may not be able to handle well.

\begin{enumerate}
\item{\textbf{seed\_violation\_handling}: What action to take if the seed airfoil violates
a constraint.  The options are 'stop' or 'warn'.}
\item{\textbf{min\_thickness}: The minimum allowable max thickness of designs, as a
fraction of the chord (for example 0.08 means 8\% thickness).  A minimum allowable
thickness is usually required for structural reasons.  If you don't want to impose this
constraint, just set it to 0.}
\item{\textbf{moment\_constraint\_type}: How to implement the pitching moment constraint.
Either 'none', 'specify', or 'use\_seed'.  If 'none', then no pitching moment constraint
is applied.  If 'specify', then the user specifies the minimum allowable moment using the 
min\_moment constraint.  If 'use\_seed', then the minimum moment from the seed airfoil 
over the specified operating points will be taken as the minimum allowable.}
\item{\textbf{min\_moment}: The minimum allowable pitching moment coefficient 
A larger negative pitching moment will require a 
larger tail.  For flying-wing airfoils, this is usually set to 0.  This input is only used
if moment\_constraint\_type = 'specify'.  If you don't want to 
impose this constraint, set moment\_constraint\_type to 'none'.  If you want the minimum
pitching moment to be no higher than the seed airfoil's, set moment\_constraint\_type to
'use\_seed'.} 
\item{\textbf{min\_te\_angle}: The minimum trailing edge wedge angle, in degrees.  Without
this constraint, the optimizer may converge on airfoils with a very thin cusped aft
section, which is challenging to build and presents structural problems.  If you don't
want to impose this constraint, just set it to 0.}
\item{\textbf{check\_curvature}: Whether to check the curvature for applying a curvature
reversals constraint.  If you don't want to apply a curvature reversals constraint, set
this to .false., because checking the curvature adds computational cost.  Set this to
.true. if you wish to specify a constraint on the number of curvature reversals.}
\item{\textbf{max\_curv\_reverse}: Maximum allowed number of curvature reversals within a
specified curvature tolerance.  If you wish to use this constraint, you must enable
check\_curvature.}
\item{\textbf{curv\_threshhold}: Curvature thresshold for detecting reversals.  The local
curvature can be thought of as 1/r, where r is the local radius.  So, for instance, a
curvature threshhold of 0.2 means that any reversals with a radius greater than 5.0 will
be ignored.  If you make this value too small, it may be very hard for the optimizer to
come up with feasible designs.  In my testing, a threshhold of 0.2 seems to work fairly
well, but you may need to increase pso\_feasible\_init\_attempts and
simplex\_feasible\_init\_attempts so that most of the initial designs are feasible.}
\item{\textbf{symmetrical} (.true. or .false.): Whether to only generate symmetrical 
airfoils.  If true, only the top surface of the seed airfoil will be modified, and the
bottom surface will be replaced by a mirrored version of the top surface. This option is
useful if you want to ensure that the final optimized airfoil is symmetrical.  The seed 
airfoil does not need
to be symmetrical to use this option (but note that XoptFoil will actually be using a
symmetrical version of the seed airfoil instead of the actual one).}
\end{enumerate}

\subsection{particle\_swarm\_options namelist}

This namelist specifies options for particle swarm optimization, which are used if
search\_type = `global' or search\_type = `global\_and\_local' and global\_search =
`particle\_swarm'.

\begin{enumerate}
\item{\textbf{pop}: Number of particles.  These are randomly initialized throughout the
design space and then use swarming behavior to converge on the global optimimum.  Search
cost increases linearly with population size.  Usually, around 40 or 50 particles seems to
work well.}
\item{\textbf{pso\_speed\_tol}: Tolerance in particle speed before signaling a stopping
condition.  When every particle's speed is less than this fraction of the initial speed,
the particle swarm optimization process is considered converged. The default value is
0.005, or 0.5\% of the initial speed.}
\item{\textbf{pso\_convergence\_profile}: Either `standard' or `exhaustive'.  This setting
adjusts some parameters in the particle swarm algorithm which affect how quickly the
optimization converges.  The standard profile works well and converges quickly.  The
exhaustive profile usually is able to find a little bit better designs than the standard
profile, but it also takes much longer to converge (e.g., 700 iterations instead of 150).}
\item{\textbf{pso\_maxit}: Maximum number of iterations allowed before the particle swarm
optimization is stopped.  Ensures that the particle swarm optimization does not go on for
too long before the stopping condition is met. The default value is 300.  Using the 
`standard' convergence profile, the optimization will usually converge in less than 300 
iterations.  However, if you use the `exhaustive' profile, this value should be increased
to 700 or 800 to ensure it doesn't stop early.}
\item{\textbf{pso\_feasible\_init} (.true. or .false.): If true, attempts to place initial
designs in the feasible space.  If an initial design violates constraints, a large penalty
value will be added to the objective function.  pso\_feasible\_init will re-randomize
designs that initially have too large an objective function value.  A maximum of 100 
attempts is allowed to make each design feasible. This process increases the overall 
expense of the optimization but also improves the results because more of the initial 
designs will be good ones.}
\item{\textbf{pso\_feasible\_limit}: The objective function value below which initial
designs are considered feasible.  Designs that violate constraints are added a penalty
value on the order of $10^6$, so the feasible limit should be large but at least an order
of magnitude smaller than $10^6$.  The default value is 50,000.}
\item{\textbf{pso\_feasible\_init\_attempts}: Number of attempts to try to make an initial
design feasible.  The default is 100, but you may want to increase this if you are using
tight constraints (especially the curvature reversals constraint).}
\end{enumerate}

\subsection{simplex\_options namelist}

This namelist specifies options for simplex optimization, which are used if
search\_type = `local' or search\_type = `global\_and\_local' and local\_search =
`simplex'.

\begin{enumerate}
\item{\textbf{simplex\_tol}: Tolerance in simplex diameter before stopping.}
\item{\textbf{simplex\_maxit}: Maximum number of iterations allowed before the simplex
optimization is stopped.  Ensures that the simplex optimization does not go on for
too long before the stopping condition is met.}
\end{enumerate}

\subsection{xfoil\_run\_options namelist}

This namelist specifies options that control how XFoil calculations are run.

\begin{enumerate}
\item{\textbf{ncrit}: Transition-triggering parameter.  A higher number represents cleaner
(less turbulent) freestream conditions.  9 is the default.}
\item{\textbf{xtript}: Top-surface trip location for turbulent transition.  For free
transition, set this to 1.  1 is the default.}
\item{\textbf{xtripb}: Bottom-surface trip location for turbulent transition.  For free
transition, set this to 1.  1 is the default.}
\item{\textbf{viscous\_mode} (.true. or .false.): Whether to run XFoil boundary layer
calculations.  The default is .true., which is recommended.  If it is .false., the pressure
drag is used as the drag coefficient, which isn't very meaningful for real designs.}
\item{\textbf{silent\_mode} (.true. or .false.): If .true., suppresses XFoil screen writes.
This is desirable for optimization, because thousands of designs are tested and there is
not time to read the XFoil notifications anyway.  For running XFoil alone without
optimizing, you may want to see the XFoil screen writes.  .true. is the default.}
\item{\textbf{bl\_maxit}: Max number of iterations for viscous calculations.  100 is the
default.}
\item{\textbf{vaccel}: XFoil viscous convergence acceleration parameter.  0.01 is the
default.}
\item{\textbf{fix\_unconverged} (.true. or .false.): Whether to try to fix
unconverged operating points by reinitializing the boundary layer at a different operating
condition.  .true. is the default.  This will make the optimization a little more
expensive, but it helps to prevent good designs from getting penalized by
non-convergence.}
\item{\textbf{reinitialize} (.true. or .false.): Whether to reinitialize the boundary
layer at each consecutive operating point.  .true. is the default, which seems to result 
in a more robust optimization, but you may want to experiment with this.}
\end{enumerate}

\subsection{xfoil\_paneling\_options namelist}

This namelist controls how XFoil smooths airfoil paneling before running aerodynamic
calculations.  The options are the same as the PPAR menu in XFoil.

\begin{enumerate}
\item{\textbf{npan}: Number of airfoil panels.}
\item{\textbf{cvpar}: Panel bunching parameter.  Increasing this number will cause more
panels to be bunched in regions of high curvature. 1.0 is the default.}
\item{\textbf{cterat}: Ratio of trailing-edge to leading-edge panel density.  0.15 is the
default.}
\item{\textbf{ctrrat}: Ratio of regular panel density to refined-area panel density.  0.2
is the default.}
\item{\textbf{xsref1}: Left top side refinement limit.  1.0 is the default, which means
there is no refined area on the top side.}
\item{\textbf{xsref2}: Right top side refinement limit.  1.0 is the default.}
\item{\textbf{xpref1}: Left bottom side refinement limit.  1.0 is the default, which means
there is no refined area on the top side.}
\item{\textbf{xpref2}: Right bottom side refinement limit.  1.0 is the default.}
\end{enumerate}

\subsection{matchfoil\_options namelist}

This namelist is used for testing new optimization algorithms and parameterization
schemes.  Instead of running an aerodynamic optimization (which is expensive), this
namelist allows you to try to match the seed airfoil shape to another known airfoil shape,
and XFoil is never run.  It's about a million times faster than aerodynamic optimization
(rough estimate), which makes it good for testing things.

\begin{enumerate}
\item{\textbf{match\_foils} (.true. or .false.): If .true., the optimizer will try to
match the seed airfoil shape to another known airfoil shape, and no aerodynamic
calculations will be performed.  .false. is the default.}
\item{\textbf{match\_foil\_file}: Text string specifying the file name of the airfoil
which is to be matched by the optimizer.  The coordinates must be given in XFoil format,
meaning the first line can be a label (but it doesn't have to be), and then the x and y
coordinates must follow in two columns arranged in a single loop beginning and ending at
the trailing edge. If the file is not in the working directory, you must include the
relative path.}
\end{enumerate}

\section{Input file for xfoil\_only}

The xfoil\_only tool uses its own namelist input file.  It is the same as the input file
for xoptfoil, except that only the following namelists are used:

\begin{itemize}
\item{airfoil\_to\_load}
\item{operating\_conditions}
\item{xfoil\_run\_options}
\item{xfoil\_paneling\_options}
\end{itemize}

The namelist called airfoil\_to\_load has a single input, called \textbf{airfoil\_file}.
Enter the name (or relative path, if it's not in the working directory) of the airfoil 
coordinates file to load here.  The operating\_conditions namelist is the same as the one
for xoptfoil, except that the \textbf{optimization\_type} and \textbf{weighting} inputs
are not used. The other namelists are the same as in the input file for xoptfoil.

\section{Output files}\label{sec:output_files}

The following files are written out by the program:

\begin{enumerate}
\item{\textbf{optfoil.dat}: The coordinates of the final optimized design.}
\item{\textbf{optfoil\_performance\_summary.dat}: Aerodynamic data for the final optimized
design at the requested operating points.}
\item{\textbf{optfoil\_design\_coordinates.dat}: Airfoil coordinates for each
progressively better design found during the optimization.  Only written if the option
write\_designs is set to .true.  Each new design is identified under one of the ``zone
t=...'' labels.  You can plot these using the design visualizer tool. See section 
\ref{sec:design_visualizer} for more information.}
\item{\textbf{optfoil\_design\_polars.dat}: Airfoil polars (lift and drag coefficients) at
the requested operating points for each progressively better design found during the 
optimization.  Only written if the option write\_designs is set to .true.  Each new design
is identified under one of the ``zone t=...'' labels.  You can plot these using the design
visualizer tool. See section \ref{sec:design_visualizer} for more information.}
\end{enumerate}

The prefix of each of these files can be changed from the default ``optfoil'' to something
else via the second command-line option when xoptfoil is run. See section
\ref{sec:running} for more information on running xoptfoil with command-line options.

\section{Using the design visualizer}\label{sec:design_visualizer}

A Python script called design\_visualizer.py is provided to plot and animate airfoil
designs generated during the optimization process.  In order to use this tool, it is
necessary to have Python with matplotlib and numpy installed.  Python 3 with matplotlib
version 1.5.0 or higher is recommended.  On Windows, the easiest
way to achieve this is to install one of the free packaged Python distributions.  The
Anaconda Python distribution is recommended because it is easy to install and includes
up-to-date versions of all the required packages.  Anaconda can be downloaded here:
\url{https://www.continuum.io/downloads}.

When installing Anaconda (or whichever Python distribution is chosen), be sure to check
the option to add Python to your system path.  That way, Python can be run from the
command prompt.  If you are not using Anaconda and the option to modify the system path is
not offered, you should manually add the appropriate directory with the python executable
to your path, following a similar method to the one described in Section
\ref{sec:compile_windows} where MinGW was added to the path.

If you are using Linux, it is likely that you already have Python installed on your
system.  However, matplotlib may not be installed by default; if so, install it from your
distro's software repository.  Note that there may be different packages for matplotlib
and numpy depending on whether you are using Python 2 or Python 3. 
For example, on Ubuntu and Linux Mint, Python 2 is installed by default, and the
matplotlib package for Python 3 is called python3-matplotlib.   Again, for best results,
matplotlib version 1.5.0 or higher is recommended.

Before running the design visualizer, you must first run an optimization with the option
write\_designs set to .true. Next, in the same directory, run the design visualizer using
the following command:

\begin{verbatim}
python design_visualizer.py
\end{verbatim}

\noindent Or, if you specified a different output file prefix when you ran xoptfoil, the 
same prefix should be passed to design\_visualizer.py like so:

\begin{verbatim}
python design_visualizer.py prefix
\end{verbatim}

Note that design\_visualizer.py must be present in the directory where you try to run it,
or it must be in your system path.  The design visualizer is menu based.  When it is run, 
it first attempts to read the coordinate and polar files generated by xoptfoil, and then
a menu is presented. The prompt should look like the following:

\begin{verbatim}
Reading airfoil coordinates from file optfoil_design_coordinates.dat...
Found 52 airfoil coordinates plus seed airfoil.
Reading airfoil polars from file optfoil_design_polars.dat...
Found 52 airfoil polars plus seed airfoil.

Options:
[0] Exit
[1] Plot a specific design
[2] Animate all designs
[3] Change plotting options

Enter a choice [0-3]: 
\end{verbatim}

Option 1 will offer you the choice to plot any of the airfoils and polars that were
generated during the optimization.  Option 2 creates the same plot, but it loops through
all the designs and displays them in sequence.  Plotting options can be changed by
selecting option 3.  Available plotting options include manual axis bounds for the plots,
turning on/off various plot elements (such as the seed airfoil, the coordinate plot, and
the polar plot), and line colors.  The plotting options menu also allows animation frames 
to be saved.
When this option is enabled and then option 2 is selected from the main menu, each frame
of the animation will be saved to a PNG image file, which can later be combined together
into a movie by a separate application such as ffmpeg.  By default, the plots generated 
will look similar to the one shown in Fig. \ref{fig:design_visualizer_plot}.  However, as
mentioned previously, the different plot elements can be turned on or off via the plot 
options menu. To quit the design visualizer tool, enter 0 at the main menu.

\begin{figure}
\centering
  \includegraphics[width=0.75\textwidth]{manual_figs/design_visualizer_plot.png}
\caption{Plot generated by design\_visualizer.py.}
\label{fig:design_visualizer_plot}
\end{figure}

A future development goal is to modify the design visualizer so that it can be run
concurrently with xoptfoil and display the optimization progress as it occurs.  The user
guide will be updated when that feature is implemented.

\end{document}
